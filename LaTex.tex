\documentclass[12pt,a4paper]{article}
\usepackage[utf8]{inputenc}
\usepackage[russian]{babel}
\usepackage[T2A]{fontenc}
\usepackage{amsmath,amssymb,amsthm}
\usepackage{algorithm}
\usepackage[noend]{algpseudocode}
\usepackage{geometry}
\usepackage{listings}
\usepackage{color}

\geometry{
    a4paper,
    left=30mm,
    right=15mm,
    top=20mm,
    bottom=20mm
}

\lstset{
    language=Python,
    basicstyle=\footnotesize\ttfamily,
    breaklines=true,
    showstringspaces=false
}

\begin{document}

\begin{titlepage}
    \centering
    {\large\textbf{Федеральное государственное автономное образовательное учреждение высшего образования}}\\[0.5cm]
    {\large\textbf{Балтийский федеральный университет имени Иммануила Канта}}\\[0.5cm]
    {\large Высшая школа компьютерных наук и искусственного интеллекта}\\[3cm]
    {\Large\textbf{ОТЧЕТ}}\\[1cm]
    {\Large по учебной практике (учебно-лабораторной)}\\[2cm]
    \begin{flushright}
        \large
        \textbf{Выполнил:}\\
        студент \underline{Мостовской Н.В.}\\[0.5cm]
        группы \underline{05-КБ-22-О}\\[0.5cm]
        направление подготовки 10.05.01 <<Компьютерная безопасность>>\\[2cm]
        \textbf{Руководитель практики:}\\
        ст. преподаватель \underline{Глинчиков К.Е.}\\[2cm]
    \end{flushright}
    \begin{flushright}
        Работа выполнена: <<\underline{\hspace{1cm}}>> \underline{\hspace{3cm}} 2025 г.\\[1cm]
        Работа защищена с оценкой: \underline{\hspace{6cm}}\\[1cm]
        <<\underline{\hspace{1cm}}>> \underline{\hspace{3cm}} 2025 г.\\
    \end{flushright}
    \vfill
    {\large Калининград 2025}
\end{titlepage}

\newpage

\section*{Цель работы}
Изучить и реализовать алгоритм Ленстры факторизации чисел с помощью эллиптических кривых (ECM). Сравнить производительность с функцией \texttt{factor()} в системе SageMath.

\section*{Описание алгоритма}
Алгоритм Ленстры использует арифметику на эллиптических кривых над кольцом вычетов по модулю $N$. При операциях с точками может возникнуть деление на число, не взаимно простое с $N$ — это и есть делитель. В отличие от других алгоритмов, ECM эффективно работает на числах с малыми простыми делителями.

\section*{Описание машины}
Работа выполнена в браузерной версии SageMath. ОС: Windows 10, ОЗУ: 8 ГБ, процессор: Intel Core i5.

\section*{Псевдокод}
\begin{algorithm}[H]
\caption{ECM – одна попытка}
\begin{algorithmic}[1]
\State Случайно выбираем параметры $x, y, a \in \mathbb{Z}_N$
\State Строим эллиптическую кривую $E$ над $\mathbb{Z}_N$ по формуле 
\[
    E: y^2 \equiv x^3 + ax + b \pmod{N}
\]
где $b \equiv y^2 - x^3 - a x \pmod{N}$. Кривая используется для вычислений в группе точек, определённых по модулю $N$.
\State Проверяем, лежит ли точка $(x, y)$ на кривой (т.е. удовлетворяет ли уравнению кривой)
\State Вычисляем $k = \text{LCM}(2, ..., B)$
\State Пробуем вычислить $kP$ на кривой
\If{возникает ошибка деления}
  \State Делитель найден
\end{algorithmic}
\end{algorithm}


\section*{Исходный код программы}
\begin{lstlisting}
from random import randint
import time

def Miller_Rabbin(n, k=20):
    r = 0
    d = n - 1
    while d % 2 == 0:
        d //= 2
        r += 1
    
    if n in (2, 3):
        return True
    
    for _ in range(k):
        a = randint(2, n - 2)
        x = pow(a, d, n)
        if x in (1, n - 1):
            continue
        for _ in range(r - 1):
            x = pow(x, 2, n)
            if x == n - 1:
                break
        else:
            return False
    return True


def ec_add(P, Q, a, N):
    if P == "O":
        return Q
    if Q == "O":
        return P

    x1, y1 = P
    x2, y2 = Q

    if x1 == x2 and (y1 + y2) % N == 0:
        return "O"

    if P != Q:
        num = (y2 - y1) % N
        den = (x2 - x1) % N
    else:
        num = (3 * x1 * x1 + a) % N
        den = (2 * y1) % N

    g = gcd(den, N)
    if g != 1 and g != N:
        raise ZeroDivisionError(g)

    try:
        inv = pow(den, -1, N)
    except ValueError:
        raise ZeroDivisionError(gcd(den, N))

    lam = (num * inv) % N
    x3 = (lam * lam - x1 - x2) % N
    y3 = (lam * (x1 - x3) - y1) % N
    return (x3, y3)


def ec_mul(k, P, a, N):
    R = "O"
    while k:
        if k % 2 == 1:
            R = ec_add(R, P, a, N)
        P = ec_add(P, P, a, N)
        k //= 2
    return R


def ecm_wiki(N, B, tries):
    for _ in range(tries):
        x = randint(1, N - 1)
        y = randint(1, N - 1)
        a = randint(1, N - 1)
        b = (y * y - x * x * x - a * x) % N

        if (y * y - (x * x * x + a * x + b)) % N != 0:
            continue

        P = (x, y)
        k = 1
        for i in range(2, B + 1):
            k = lcm(k, i)

        try:
            ec_mul(k, P, a, N)
        except ZeroDivisionError as e:
            d = e.args[0]
            if 1 < d < N:
                return d
    return None


def ecm_full_factor(N, B, tries):
    factors = []

    def recurse(n):
        if n == 1:
            return
        if Miller_Rabbin(n):
            factors.append(n)
            return

        d = ecm_wiki(n, B, tries)
        if d is None or d == n:
            factors.append(n)
            return
        recurse(d)
        recurse(n // d)

    recurse(N)
    return sorted(factors)

N = 2566694586072166993
start = time.time()
factors = ecm_full_factor(N, B=300, tries=250)
end = time.time()

print(f"Prime factors: {factors}")
print(f"Time: {end - start:.4f} sec")
\end{lstlisting}

\section*{Результаты сравнения}
\begin{tabular}{|c|c|c|}
\hline
Метод & Множители & Время (сек) \\
\hline
ECM & [487, 263521, 20000000359] & 0.14 \\
\hline
Sage \texttt{factor()} & [487, 263521, 20000000359] & 0.01 \\
\hline
\end{tabular}

\section*{Выводы}
Алгоритм Ленстры успешно реализован и протестирован. Он корректно находит делители и пригоден для чисел, устойчивых к другим методам. Однако по производительности уступает встроенной функции \texttt{factor()} в SageMath.

\newpage

\section*{Список литературы}

\begin{enumerate}
    \item Lenstra A.K. "Factoring integers with elliptic curves". *Annals of Mathematics*, 1987.\\
    \url{https://wstein.org/edu/124/lenstra/lenstra.pdf}
    \item Википедия. Факторизация с помощью эллиптических кривых.\\
    \url{https://ru.wikipedia.org/wiki/Факторизация_с_помощью_эллиптических_кривых}
\end{enumerate}


\end{document}
